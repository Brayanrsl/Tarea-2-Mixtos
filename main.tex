\documentclass{article}
\usepackage{fancyhdr,lipsum}
\pagestyle{fancy}
\fancyhead[C]{\rule{.5\textwidth}{4\baselineskip}}% Add something BIG in the header
\setlength{\headheight}{52pt}
\usepackage[utf8]{inputenc}
\usepackage[spanish]{babel}
\usepackage[T1]{fontenc}
\usepackage{amsmath}
\usepackage{amsfonts}
\usepackage{amssymb}
\usepackage{amsthm}
\usepackage{stmaryrd}
\usepackage[mathscr]{euscript}
% Cajas
\usepackage{tcolorbox}
\usepackage{tikz}
% Comandos
\newtheorem{teo}{Teorema}
\newtheorem{prop}{Proposición}
\newtheorem{pro}{Problema}
\newcommand{\autor}{\textbf{Brayan Sandoval}}
\newcommand{\asignatura}{\textbf{Álgebra II}}
\newcommand{\tarea}{\textbf{Listado 2}}
\newcommand{\fecha}{\textbf{\today}}
\newcommand{\bs}{\boldsymbol}
\newcommand{\dx}{\textup{d}x}
\newcommand{\dt}{\textup{d}t}
\newcommand{\DG}{\textup{DG}}
\newcommand{\Th}{\mathscr{T}_{h}}
\newcommand{\Hdiv}{H\left(\textup{div};\Omega\right)}
\newcommand{\Hdivt}{H(\textup{div};\mathscr{T}_{h})}
\providecommand{\Dt}[1]{\frac{\textup{d} #1}{\textup{d}t}}
\providecommand{\Dx}[1]{\frac{\textup{d} #1}{\textup{d}x}}
\providecommand{\abs}[1]{\left\lvert#1\right\rvert}
\providecommand{\norm}[1]{\left\lVert#1\right\rVert}
\providecommand{\salto}[1]{\left\llbracket#1\right\rrbracket}
\providecommand{\prom}[1]{\left \{\!\left \{#1\right \}\!\right \}}
\providecommand{\PI}[2]{\left\langle #1,#2  \right\rangle}
\providecommand{\Pii}[2]{\left( #1,#2  \right)}
\renewcommand{\theequation}{\roman{equation}}
%texto
\usepackage{lipsum} % Genera texto aleatorio
\renewcommand*{\familydefault}{\sfdefault} % Letra mas bonita
% Figuras
\usepackage{graphicx}
% Geometría
\usepackage[left= 2 cm, right = 2 cm, top = 2 cm, bottom = 2 cm]{geometry}
\usepackage{lastpage}
% Color
\usepackage{xcolor}
\definecolor{azul}{RGB}{10,10,115}
\definecolor{amarillo}{RGB}{255,204,0}
\definecolor{rojo}{RGB}{247,0,30}
% Encabezado
\usepackage{fancyhdr}
\pagestyle{fancy}
\renewcommand{\headrulewidth}{4pt} %Aumentar grosor linea encabezado
\let\oldheadrule\headrule
\renewcommand{\headrule}{\color{azul}\oldheadrule}
\renewcommand{\footrulewidth}{4pt} %Aumentar grosor linea pie de pagina
\let\oldfootrule\footrule
\renewcommand{\footrule}{\color{azul}\oldfootrule}
\rhead{\color{azul}\autor}
\chead{\color{azul}\tarea}
\lhead{\color{azul}\asignatura}
\rfoot{\color{azul} \textbf{Pág. \thepage\ - \pageref{LastPage}}}
\cfoot{}
\lfoot{\color{azul}\fecha}
% Titulo
\title{\color{azul}\textbf{Álgebra II}\\
	\textbf{Listado 2}}
\author{\color{azul}\autor}
\date{\color{azul}\fecha}
\begin{document}
\begin{pro}
    Sean $\alpha,\beta\in \mathbb{R}$ un parámetro con el cual se define $A := \begin{pmatrix}
            \alpha & \beta  & 0      & \beta  \\
            \beta  & \alpha & \beta  & 0      \\
            0      & \beta  & \alpha & \beta  \\
            \beta  & 0      & \beta  & \alpha
        \end{pmatrix}$. Determine para qué valores de $\alpha$ y $\beta$, $A$ es no singular.
\end{pro}
\begin{proof}[Solución]
    Procederemos por casos:
    \begin{itemize}
        \item[$i$)] $\beta = 0$ y $\alpha \neq 0$. En este caso $A$ es una matriz triangular, entonces $\displaystyle\det\left(A\right) = \prod_{i=1}^{4}a_{ii} = \alpha^{4} \neq 0$. Por tanto, la matriz $A$ es no singular.
        \item[$ii)$] $\beta\neq 0$ y $\alpha \neq 0$. Para este caso usaremos el siguiente teorema
            \begin{tcolorbox}[title= \textbf{Teorema},colback=blue!4,colframe=blue!30!black!80,colbacktitle=blue!30!black!80]
                $A\in \mathcal{M}_{n}(\mathbb{K})$ es invertible, si y sólo si, $\det(A)\neq 0.$
            \end{tcolorbox}
            Para obtener el determinante de $A$ haremos operaciones elementales por fila
            \begin{align*}
                 & \begin{pmatrix}
                       \alpha & \beta  & 0      & \beta  \\
                       \beta  & \alpha & \beta  & 0      \\
                       0      & \beta  & \alpha & \beta  \\
                       \beta  & 0      & \beta  & \alpha
                   \end{pmatrix} \underset{\text{\Huge$\sim$}}{f_{2}\leftarrow f_{2}-f_{4}}
                \begin{pmatrix}
                    \alpha & \beta  & 0      & \beta   \\
                    0      & \alpha & 0      & -\alpha \\
                    0      & \beta  & \alpha & \beta   \\
                    \beta  & 0      & \beta  & \alpha
                \end{pmatrix} \underset{\text{\Huge$\sim$}}{f_{4}\leftarrow f_{4}-\frac{\beta}{\alpha}f_{1}}
                \begin{pmatrix}
                    \alpha & \beta                     & 0      & \beta                               \\
                    0      & \alpha                    & 0      & -\alpha                             \\
                    0      & \beta                     & \alpha & \beta                               \\
                    0      & -\frac{\beta^{2}}{\alpha} & \beta  & \frac{\alpha^{2}-\beta^{2}}{\alpha}
                \end{pmatrix} \\
                 & \underset{\text{\Huge$\sim$}}{f_{3}\leftarrow f_{3}-\frac{\beta}{\alpha}f_{2}}
                \begin{pmatrix}
                    \alpha & \beta                     & 0      & \beta                               \\
                    0      & \alpha                    & 0      & -\alpha                             \\
                    0      & 0                         & \alpha & 2\beta                              \\
                    0      & -\frac{\beta^{2}}{\alpha} & \beta  & \frac{\alpha^{2}-\beta^{2}}{\alpha}
                \end{pmatrix}
                \underset{\text{\Huge$\sim$}}{f_{4}\leftarrow f_{4}+\frac{\beta^{2}}{\alpha^{2}}f_{2}}
                \begin{pmatrix}
                    \alpha & \beta  & 0      & \beta                                \\
                    0      & \alpha & 0      & -\alpha                              \\
                    0      & 0      & \alpha & 2\beta                               \\
                    0      & 0      & \beta  & \frac{\alpha^{2}-2\beta^{2}}{\alpha}
                \end{pmatrix}                             \\
                 & \underset{\text{\Huge$\sim$}}{f_{4}\leftarrow f_{4}-\frac{\beta}{\alpha}f_{3}}
                \begin{pmatrix}
                    \alpha & \beta  & 0      & \beta                                \\
                    0      & \alpha & 0      & -\alpha                              \\
                    0      & 0      & \alpha & 2\beta                               \\
                    0      & 0      & 0      & \frac{\alpha^{2}-4\beta^{2}}{\alpha}
                \end{pmatrix}.
            \end{align*}
            Usando el hecho que la matriz obtenida después del proceso de escalonamiento es triangular superior y dado que solo realizamos operaciones elementales del tipo $f_{i}\leftarrow f_{i}+ \lambda f_{j}$, donde $\lambda\in\mathbb{R}$ y $j,i\in\{1,2,3,4\}$, se obtiene que
            \begin{align*}
                \det(A) = \det\begin{pmatrix}
                                       \alpha & \beta  & 0      & \beta                                \\
                                       0      & \alpha & 0      & -\alpha                              \\
                                       0      & 0      & \alpha & 2\beta                               \\
                                       0      & 0      & 0      & \frac{\alpha^{2}-4\beta^{2}}{\alpha}
                                   \end{pmatrix} = \alpha\cdot\alpha\cdot\alpha\cdot \frac{\alpha^{2}-(2\beta)^{2}}{\alpha} = \alpha^{2}(\alpha^{2}-(2\beta)^{2}) = \alpha^{2}(\alpha - 2\beta)(\alpha + 2\beta)
            \end{align*}
            Así, $A$ es no singular, si y sólo si, $\alpha \neq 2\beta$ y $\alpha \neq -2\beta$.
        \item[$iii)$] Si $\alpha = 0$ y $\beta\neq 0$. En este caso la matriz tiene la forma
            \begin{align*}
                A = \begin{pmatrix}
                             0     & \beta & 0     & \beta \\
                             \beta & 0     & \beta & 0     \\
                             0     & \beta & 0     & \beta \\
                             \beta & 0     & \beta & 0
                         \end{pmatrix}
            \end{align*}
            Luego, dado que las fila $1$ y $3$ son iguales, entonces $\det(A) = 0$ y por tanto $A$ es singular.
        \item[$iv)$] $\alpha = \beta = 0$. Dado que $A$ es la matriz nula, $\det(A) = 0$, por tanto $A$ es singular.
    \end{itemize}
\end{proof}
\end{document}